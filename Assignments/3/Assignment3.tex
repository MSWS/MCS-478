\documentclass{article}

\usepackage{amsmath}
\usepackage{amssymb}
\usepackage{enumitem}
\usepackage{cancel}
\usepackage{listings}
\usepackage{geometry}[margin=1in]
\usepackage{tikz}

\title{\vspace*{-4em}M/CS 478 Assignment 3}
\author{Isaac Boaz}
\newcommand{\Mod}[1]{\ (\mathrm{mod}\ #1)}

\begin{document}
\maketitle

\section*{2.17}
\begin{enumerate}[label=(\alph*)]
    \item \(11^x = 21 \text{ in } \mathbb{F}_{71}\) \\
          Using Shanks's babystep-giantstep method, let's first
          populate the table for \([1, m]\) where \(m = \lceil \sqrt{71} \rceil = 9\)
          \begin{center}
              \begin{tabular}{|c|c|c|c|c|c|c|c|c|}
                  \hline
                  11 & 50 & 53 & 15 & 23 & 40 & 14 & 12 & 61 \\
                  \hline
              \end{tabular}
          \end{center}

          Using Fermat's Little Theorem \footnote{Since 11 is a primitive root mod 71}, we can find the "bottom" cell of the table:
          \begin{align*}
              20x \equiv 1 \Mod{71} &              \\
              71                    & = 3(20) + 11 \\
              20                    & = 1(11) + 9  \\
              11                    & = 1(9) + 2   \\
              9                     & = 4(2) + 1   \\
              2                     & = 2(1) + 0
          \end{align*}
          \begin{align*}
              1       & = 9 - 4(2)              \\
                      & = 9 - 4(11 - 9)         \\
                      & = 5(9) - 4(11)          \\
                      & = 5(20 - 11) - 4(11)    \\
                      & = 5(20) - 9(11)         \\
                      & = 5(20) - 9(71 - 3(20)) \\
                      & = 32(20) - 9(71)        \\
              20^{-1} & = 32 \Mod{71}
          \end{align*}
          Finally we multiply by the inverse to find the answer:
          \begin{align*}
              21 \times 32 & = 33 \Mod{71} \\
              \times 32    & = 62 \Mod{71} \\
              \times 32    & = 67 \Mod{71} \\
              \times 32    & = 14 \Mod{71} \\
          \end{align*}
          Since 14 was in the top row of the table, and we multiplied by the inverse (ie went up)
          4 times, we know the correct cell is in the 7th column, 4th row. Thus
          \(x = 3(10) + 7 = 37\).

          Plugging this back into the original equation, we get
          \(11^{37} = 21 \Mod{71}\), which is true.
\end{enumerate}

\pagebreak

\section*{3.7}
Alice publishes her RSA public key: modulus \(N = 2038667\) and exponent \(e = 103\).
\begin{enumerate}[label=(\alph*)]
    \item Bob wants to send alice the message \(m = 892383\). What ciphertext does Bob send to Alice?
          The formula for calculating ciphertext is
          \begin{equation*}
              m^e \equiv c \Mod{N}
          \end{equation*}
          Thus, Bob simply needs to calculate
          \begin{align*}
              892383^{103} & \equiv c \Mod{2038667}     \\
              c            & \equiv 45293 \Mod{2038667}
          \end{align*}
    \item Alice knows that her modulus factors into a product of two primes, one of which is \(p = 1301\).
          Find a decryption exponent \(d\) for Alice.

          We know that \(N = 2038667 = p \cdot q\), and that \(e\) is the public exponent.
          We also know that \(d\) is the private exponent, and that \(d\) is the modular inverse of \(e\) mod \(\phi(N)\).
          We can calculate \(\phi(N)\) using the formula \(\phi(N) = (p-1)(q-1)\).
          \begin{align*}
              \phi(N) & = (1301-1)(1567-1) \\
                      & = 1300 \cdot 1566  \\
                      & = 2035800
          \end{align*}

          We can then calculate the modular inverse of \(e\) mod \(\phi(N)\) using the extended Euclidean algorithm.
          \begin{align*}
              2035800 & = 19765(103) + 5 \\
              103     & = 20(5) + 3      \\
              5       & = 1(3) + 2       \\
              3       & = 1(2) + 1
          \end{align*}
          \begin{align*}
              1        & = 3 - 2                             \\
                       & = 3 - (5 - 3)                       \\
                       & = 2(3) - 5                          \\
                       & = 2(103 - 20(5)) - 5                \\
                       & = 2(103) - 41(5)                    \\
                       & = 2(103) - 41(2035800 - 19765(103)) \\
                       & = 810367(103) - 41(2035800)         \\
              103^{-1} & = 810367 \Mod{2035800}
          \end{align*}
          Since \(ed \equiv 1 \Mod{\phi{(n)}}\), \(d = e^{-1} = 810367\).
    \item Alice receives the ciphertext \(c = 317730\) from Bob. Decrypt the message.
          The formula for calculating the plaintext is
          \begin{equation*}
              c^d \equiv m \Mod{N}
          \end{equation*}
          Thus, Alice simply needs to calculate
          \begin{align*}
              317730^{810367} & \equiv m \Mod{2038667}      \\
              m               & \equiv 514407 \Mod{2038667}
          \end{align*}
          Thus, Alice receives the message \(m = 514407\).
\end{enumerate}

\section*{3.11}
Here is an example of a proposed public key system.

Alice chooses two large primces \(p\) and \(q\) and she publishes \(N = pq\).
It is assumed that \(N\) is hard to factor. Alice chooses three random numbers
\(g, r_1, \text{ and } r_2\) modulo \(N\) and computes
\begin{equation*}
    g_1 \equiv g^{r_1(p-1)} \Mod{N} \text{ and } g_2 \equiv g^{r_2(q-1)} \Mod {N}.
\end{equation*}

Her public key is the triple \((N, g_1, g_2)\) and her private key is the pair of
primes \((p, q)\). Now Bob wants to send the message \(m\) to Alice where \(m\) is
a number modulo \(N\). He chooses two random integers \(s_1, s_2\) modulo \(N\) and computes
\begin{equation*}
    c_1 \equiv m \cdot g_1^{s_1} \Mod{N} \text{ and } c_2 \equiv m \cdot g_2^{s_2} \Mod{N}.
\end{equation*}

Bob sends the ciphertext \((c_1, c_2)\) to Alice. Alice decrypts the message using the
Chinese Remainder Theorem.

\begin{equation*}
    x \equiv c_1 \Mod{p} \text{ and } x\equiv c_2 \Mod{q}
\end{equation*}


\begin{enumerate}[label=(\alph*)]
    \item Prove that Alice's solution \(x\) is equal to Bob's plaintext \(m\).
          \begin{align*}
              c_1 & \equiv mg_1^{s_1}            \Mod{p}      \\
                  & \equiv {mg^{r_1(p-1)}}^{s_1} \Mod{p}      \\
                  & \equiv {mg^{r_1s_1(p-1)}}    \Mod{p}      \\
                  & \equiv {mg^{(p-1)}}^{r_1s_1} \Mod{p}      \\
                  & \equiv {m1}^{r_1s_1}         \Mod{p}      \\
                  & \equiv m \Mod{p}                          \\
              c_2 & \equiv mg_2^{s_2}                         \\
                  & \equiv {mg^{r_2(q-1)}}^{s_2} \Mod{q}      \\
                  & \equiv {mg^{r_2s_2(q-1)}}    \Mod{q}      \\
                  & \equiv {mg^{(q-1)}}^{r_2s_2} \Mod{q}      \\
                  & \equiv {m\cdot1}^{r_2s_2}         \Mod{q} \\
                  & \equiv m \Mod{q}
          \end{align*}
          \begin{align*}
              x \equiv c_1  \equiv m \Mod{p} \\
              x \equiv c_2  \equiv m \Mod{q}
          \end{align*}
          Since the CRT guarantees a unique solution modulo \(N\), the solution
          that Alice finds \textit{must} be equal to \(m\).
          \begin{align*}
              x & \equiv m \Mod{p} \\
              x & \equiv m \Mod{q}
          \end{align*}
    \item Since this uses the Chinese Remainder Theorem, \(m\) must be smaller than both \(p\) and \(q\),
          otherwise CRT could return \(m + xN\).

          Additionally, given the two ciphertexts \((c_1, c_2)\), the following attack is possible:
          \begin{align*}
              c_1 \cdot c_2 ^ {-1} & \equiv (m \cdot g_1^{s_1})(m\cdot g_2^{s_2})^{-1}                     \\
                                   & \equiv m^2 \cdot g_1^{s_1} \cdot g_2^{-s_2}                           \\
                                   & \equiv m^2 \cdot ({g^{r_1(p-1)}}^{s_1}) \cdot ({g^{r_2(q-1)}}^{-s_2}) \\
                                   & \equiv m^2 \cdot 1^{s_1} \cdot 1^{-s_2}                               \\
                                   & \equiv m^2
          \end{align*}
\end{enumerate}

\section*{3.13}
Alice decides to use RSA with the public key \(N = 1889570071\). In order to guard against
transmission errors, Alice has Bob encrypt his message twice, once using the encryption
exponent \(e_1 = 1021763679\) and once using the encryption exponent \(e_2 = 519424709\).
Eve intercepts the two encrypted messages
\begin{equation*}
    c_1 = 1244183534 \text{ and } c_2 = 732959706
\end{equation*}

Assume Eve also knows \(N\) and the two ecnryption exponents \(e_1, e_2\), help Eve recover
Bob's plaintext without finding a factorization of \(N\).

Since the gcd\((c_1, c_2) = 1\), Eve can calculate a soultion to
\begin{equation*}
    e_1 \cdot u + e_2 \cdot v = 1
\end{equation*}
and then use \(u, v\) to calculate
\begin{align*}
    c_1^u \cdot c_2^v & = m^{e_1\cdot u + e_2 \cdot v} \Mod{N} \\
                      & = m^{\text{gcd}(e_1, e_2)} \Mod{N}     \\
                      & = m^1
\end{align*}
\end{document}