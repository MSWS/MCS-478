\documentclass{article}

\usepackage{amsmath}
\usepackage{amssymb}
\usepackage{enumitem}

\author{Isaac Boaz}

\begin{document}

\section*{Problem 1.2}
\begin{enumerate}[label=(\alph*)]
    \item ITHINKTHATISHALLNEVERSEEABILLBOARDLOVELYASATREE
    \item LOVEISNOTLOVEWHICHALTERSWHENITALTERATIONFINDS
    \item INBAITINGAMOUSETRAPWITHCHEESEALWAYSLEAVEROOMFORTHEMOUSE 
\end{enumerate}

\section*{Problem 1.4}

\section*{Problem 1.5}
For simplicity's sake, I'll be using A, B, C, D as the alphabet.
\begin{enumerate}[label=(\alph*)]
    \item \(4! = 4 \cdot 3 \cdot 2 \cdot 1 = 24 \text{ possible substitution ciphers}\)
    \item \begin{enumerate}[label=(\roman*)]
        \item For no fixed letters, A can map to B, C, or D, so\dots
        \begin{equation*}
            3 \cdot 2 \cdot 1 = 6 \text { possible substitution ciphers that have no letters fixed}
        \end{equation*}
        \item For at least one fixed letter, we have 4 choices for the fixed letter, and then we simply deal with a 3-letter alphabet
        \begin{equation*}
            4 \cdot 3 \cdot 2 \cdot 1 = 24 \text { possible substitution ciphers that have at least one letter fixed}
        \end{equation*}
        \item For only one fixed letter, we have 4 choices for the fixed letter, and then we simply deal with a 3-letter alphabet
        without any fixed letters, so\dots
        \begin{equation*}
            4 \cdot 2 \cdot 1 = 8 \text { possible substitution ciphers that have one letter fixed}
        \end{equation*}
    \end{enumerate}
\end{enumerate}
\end{document}