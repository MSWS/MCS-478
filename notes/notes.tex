\documentclass{article}

\usepackage[tmargin=0.1in, bmargin=0.25in, lmargin = 0.1in, rmargin = 0.25in]{geometry}
\usepackage{amsmath, amssymb, amsthm}
\usepackage{enumitem}
\usepackage{multicol}
\usepackage{listings}
\usepackage[T1]{fontenc}
\usepackage{lmodern}
\usepackage{cancel}

\newcommand{\Mod}[1]{\ (\mathrm{mod}\ #1)}
\begin{document}
\small
\begin{multicols*}{2}
    \subsection*{Multiplication Principle}

    \begin{tabular}{|r|c|c|}
        \hline
                  & Repetition           & No Repitition           \\
        \hline
        Ordered   & \(n^r\)              & \(\frac{n!}{(n-r)!}\)   \\
        Unordered & \(\binom{n+r-1}{r}\) & \(\frac{n!}{r!(n-r)!}\) \\
        \hline
    \end{tabular}
    \begin{equation*}
        P(A \mid B) = \frac{P(A \cup B)}{P(B)}
    \end{equation*}

    \section*{Basic Number Theory}
    \subsection*{Euclidean Algorithm}
    % Runtime is linear with the number of digits in the input
    % Eg: 2x100 digit numbers would take ~200 steps, not a thousand
    \subsubsection*{\(ax +by = d\)}
    % Can always express the gcd of two numbers as a linear combination of them
    \begin{align*}
        \text{gcd}(6, 10) & = 2             \\
        2                 & = 10x + 6y      \\
                          & = 10(-1) + 6(2)
    \end{align*}
    \subsection*{Congruence Theory}
    \subsection*{Fast Powering Algorithm}
    \begin{multicols*}{2}
        \begin{align*}
            2^{100} \Mod{5} & \equiv 2^{64} \cdot 2^{32} \cdot 2^{4} \Mod{5} \\
                            & \equiv 1 \cdot 4 \Mod{5}                       \\
                            & \equiv 4 \Mod{5}
        \end{align*}
        \columnbreak
        \begin{align*}
            2^1 & = 2 \Mod{5} \\
            2^2 & = 4 \Mod{5} \\
            2^4 & = 1 \Mod{5} \\
        \end{align*}
    \end{multicols*}
    \subsection*{Fermat's Little Theorem}
    If \(p \nmid a \land p \text{ is prime } \implies a^{p-1} \equiv 1\Mod{p}\)
    \subsection*{Euler's Theorem}
    \(a^{\phi{(n)}} \equiv 1 \Mod{n}\)
    \subsection*{Primitve Root Theorem}
    Every prime \(p\) has a primitive root
    \section*{Cryptography}
    \subsection*{Symmetric}
    \subsubsection*{Ideal Requirements}
    \begin{enumerate}[label=\arabic*)]
        \item With key it should be easy to encrypt/decrpyt.
        \item Without key it should be difficult to encrypt/decrypt.
        \item Even with lots of plaintexts <-> combinations, it should be difficult to find the key.
        \item Choosen plaintext attack: Attacker can choose plaintexts and see the corresponding ciphertexts.
    \end{enumerate}
    \subsubsection*{Multiplication} vulnerable to plaintext <-> cyphertext attacks
    \begin{equation*}
        E(x) = x \cdot k \Mod{n}
    \end{equation*}
    \columnbreak
    \section*{Primality Testing}
    \subsection*{Miller-Rabin Test} builds off of Fermat's Test
    \begin{description}
        \item[Probabilistic] \(\rightarrow\) try 100 candidates (to be witnesses)
        \item[If \(n\) is composite] overwhelmingly likely to find a witness
    \end{description}
    \begin{equation*}
        \text{If } n \text{ is prime}, a^{n-1} \equiv 1 \Mod{n}
    \end{equation*}
    \begin{enumerate}
        \item Make a table where \(n-1 = 2^kq, q \in \text{Odd}\)
              \begin{equation*}
                  a^q, a^{2q}, a^{4q}, \ldots, a^{2^{k-1}q}
              \end{equation*}
        \item Either first number is 1 (probably prime), or one of the numbers
              is -1
        \item Last number \textbf{has} to be 1 (we passed Fermat's test)
        \item If second to last number is not 1, then \(n\) is composite
        \item Consider the first term in the sequence congruent to 1. If the
              preceding term is \textit{not} congruent to -1, then \(n\) is
              composite.
    \end{enumerate}
    \begin{multicols*}{2}
        \begin{minipage}{\columnwidth}
            \begin{align*}
                n = 252601, n-1 = 2^3 \cdot  31575 \\
                a = 85132
            \end{align*}
            \begin{align*}
                a^{31575}         & \equiv 191102 \Mod {n} \\
                a^{2 \cdot 31575} & \equiv 184829 \Mod {n} \\
                a^{4 \cdot 31575} & \equiv 1 \Mod {n}
            \end{align*}
            \textit{Conclusion:} \(n\) \textit{is \textbf{composite}.}
            \begin{align*}
                n = 104717, n-1 = 2^2 \cdot 26179 \\
                a = 96152
            \end{align*}
            \begin{align*}
                a^{26179} & \equiv 1 \Mod {n}
            \end{align*}
            \textit{Conclusion:} \(n\) \textit{is \textbf{probably prime}}.
        \end{minipage}
        \columnbreak
        \begin{minipage}{\columnwidth}
            \begin{align*}
                n = 3057601, n-1 = 2^6 \cdot 47775 \\
                a = 99908
            \end{align*}
            \begin{align*}
                a^{47775}           & \equiv 1193206 \Mod {n} \\
                a^{2 \cdot 47775}   & \equiv 2286397 \Mod {n} \\
                a^{2^2 \cdot 47775} & \equiv 235899 \Mod {n}  \\
                a^{2^3 \cdot 47775} & \equiv 1 \Mod {n}
            \end{align*}
            \textit{Conclusion:} \(n\) \textit{is \textbf{composite}.}
            \begin{align*}
                n = 577757, n-1 = 2^2 \cdot 144439 \\
                a = 314997
            \end{align*}
            \begin{align*}
                a^{144439}         & \equiv 373220 \Mod {n} \\
                a^{2 \cdot 144439} & \equiv -1 \Mod {n}
            \end{align*}
            \textit{Conclusion:} \(n\) \textit{is \textbf{probably prime}}.
        \end{minipage}
    \end{multicols*}
    % Spits out numbers that overwhelmingly likely to be prime
    % Try a candidate number, each time you try a candidate
    % number, you have a 75% chance of getting a witness
    % for the compositeness of the number
    \subsection*{Shanks's Algorithm}
    \begin{multicols*}{2}
        \begin{align*}
            g^x \equiv h \Mod{p}                      \\
            p \text{ prime}                           \\
            g \text{ primitive root}                  \\
            N = p - 1                                 \\
            \text{Solve for } x: g^x \equiv h \Mod{p} \\[1em]
            n = \lceil \sqrt{N} \rceil
        \end{align*}
        \columnbreak
        \begin{align*} g, g^2, g^3,       & \ldots, g^n          \\
               g^{-n}, g^{-n+1},  & \ldots, g^{-1}       \\
               hg^{-n}, hg^{-2n}, & \ldots, hg^{-(n-1)n} \\
        \end{align*}
    \end{multicols*}
    \vspace*{-8em}
    \begin{align*}
        p = 101                                     \\
        g = 2                                       \\
        \text{Once you get } hg^{-jn} = g^i \Mod{p} \\
        h = g^{i + jn} \Mod{p}                      \\
    \end{align*}
    \subsubsection*{Pollart's Rho Algorithm}
    An improvement only in space.
    \subsubsection*{Randomized Algorithm}
    Work out random powers of \(g\), and random powers of \(hg\).
    Compare the two lists, and if you find a match, you can solve for \(x\).
    \begin{align*}
        g^x \equiv h \Mod{p} \\
        h = g^x \Mod{p}      \\
        h = g^{x + kn} \Mod{p}
    \end{align*}
\end{multicols*}
\pagebreak
\begin{multicols}{2}
    \section*{RSA}
    \subsection*{Public Key Cryptography}
    \begin{align*}
        p, q      & \text{ large prime numbers} \sim 2^{1000} \\
        N         & = pq                                      \\
        \phi{(N)} & = (p-1)(q-1)
    \end{align*}
    \begin{itemize}
        \item Encryption exponent \(e\) s.t gcd\((e, \phi(n) =1)\)
        \item Decryption exponent \(d\) s.t \(ed \equiv 1 \Mod{\phi(N)}\)
        \item Encrypting: \(m \rightarrow m^e \equiv c \Mod{N}\)
        \item Decrypting: \(c \rightarrow c^d \equiv m^{ed} \equiv m \Mod{N}\)
    \end{itemize}

    \subsection*{Digital Signatures}
    Private signing key \(d\), public verification key \(e\)
    \begin{align*}
        \text{Signer (Sam) } S \equiv D^d \Mod{N} \\
        \text{Verifier (Victor) } D \equiv S^e \Mod{N}
    \end{align*}

    \section*{Size of input}
    Given \(N\), the size of the input is \(\log_2{N}\) bits.

    \section*{Group Theory}
    \subsection*{Multiplicative Group Mod p}
    \begin{equation*}
        \mathbf{F}_p^x = \{1, 2, 3, \ldots, p-1\}, \text{ under multiplication modulo } p.
    \end{equation*}
    A group \(G\) is a set, together with a rule for combining ordered
    pairs of elements to yield another element in the same set.

    \begin{enumerate}[label=(\Roman*)]
        \item \(e \times a = a \times e = a\) for all \(a \in G\) % Identity
        \item \(a^{-1} \times a = a^{-1} \times a = e\) for all \(a \in G\) % Inverse
        \item \(a \times (b \times c) = (a \times b) \times c\) for all \(a, b, c \in G\) % Associative
    \end{enumerate}
    All the numbers in the set must be coprime to \(p\) to form a cyclic group.
    \begin{align*}
        13^{-1}   & \equiv 1 \Mod{17} \\
        13x + 17y & = 1               \\
    \end{align*}
    \begin{align*}
        17 & = 1(13) + 4 \\
        13 & = 3(4) + 1  \\
    \end{align*}
    \begin{align*}
        1 & = 13 - 3(4)       \\
          & = 13 - 3(17 - 13) \\
          & = 4(13) - 3(17)   \\
          & \equiv 4 \Mod{17}
    \end{align*}

    \columnbreak

    \section*{Diffie-Hellman}
    \subsection*{Key Exchange}
    \(p\) large prime \((\sim 2^{1000})\) \\
    \(g \in \mathbf{F}_p^x\) has \underline{large prime order} in \(\mathbf{F}_p^x\)
    \begin{enumerate}
        \item Alice picks \(a\) and sends \(A \equiv g^a \Mod{p}\) to Bob
        \item Bob picks \(b\) and sends \(B \equiv g^b \Mod{p}\) to Alice
        \item Alice computes \(B^a \equiv (g^b)^a \equiv g^{ab} \Mod{p}\)
        \item Bob computes \(A^b \equiv (g^a)^b \equiv g^{ab} \Mod{p}\)
    \end{enumerate}

    \section*{El Gamel}
    \begin{itemize}
        \item Alice picks \(a\) and sends \(A \equiv g^a \Mod{p}\) to Bob
        \item Bob chooses \(k\) and computes \(c_1 \equiv g^k, g_2 \equiv mA^k\)
        \item Alice receives \(c_1, c_2\) and computes \(m \equiv c_2(c_1^{-a}) \Mod{p}\)
              \begin{align*}
                   & c_2c_2^{-a}               \\
                   & \equiv mg^{ak} (g^k)^{-a} \\
                   & \equiv m \Mod{p}
              \end{align*}
    \end{itemize}

    \subsection*{Digital Signatures}
    \begin{itemize}
        \item Samantha chooses \(a\) (secret), computes \(A \equiv g^a \Mod{p}\)
        \item Also chooses \(k\) coprime to \(p-1\), ie gcd\((k, p-1) = 1\)
              \begin{align*}
                  S_1 & = g^k \Mod{p}                \\
                  S_2 & = (D - aS_1)k^{-1} \Mod{p-1}
              \end{align*}
              \subsubsection*{Verification}
              \begin{align*}
                  A^{S_1}S_1^{S_2} & \equiv g^D \Mod{p} \\
              \end{align*}
              \begin{align*}
                  A_1^{S_1}S_1^{S_2} & = g^{aS_1}g^{\cancel{k}(D-aS_1) \cancel{k^{-1}}} \\
                                     & = g^{aS_1}g^{D-aS_1} = g^D
              \end{align*}
    \end{itemize}
\end{multicols}
\end{document}